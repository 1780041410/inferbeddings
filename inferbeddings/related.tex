% !TeX spellcheck = en_GB

\section{Related Works} \label{sec:related}

%
Research on leveraging rules when learning Knowledge Graph embeddings has been deeply important for new developments in the field of Knowledge Base completion.
%
In~\cite{rocktaschel14low,DBLP:conf/naacl/RocktaschelSR15} authors provide a framework for jointly maximizing the probability of observed facts and propositionalised First-Order logic rules.
%
In~\cite{DBLP:conf/ijcai/WangWG15} authors show how different types of rules can be included after training the model by using Integer Linear Programming.
%
In~\cite{DBLP:conf/ijcai/WangC16} authors propose a method for embedding facts and rules using matrix factorization.
%
However, all these approaches ground the rules in the training data, limiting their scalability towards large Knowledge Bases containing a large number of entities.
%
As mentioned in~\cite{DBLP:conf/emnlp/DemeesterRR16}, such problems provide an important motivation for \emph{lifted} rule injections methods that do not rely on the grounding of logic formulas.
%
In~\cite{DBLP:conf/cikm/Wei0LQST15} authors try to work around this problem by reasoning on a filtered subset of grounded facts.
%

%
In~\cite{DBLP:conf/aaai/WuSYLZZ15}, authors propose using the Path Ranking Algorithm~\cite{DBLP:conf/emnlp/LaoMC11} for capturing long-range interactions between entities and modelling these by defining an additional loss term.
%
The model in this paper differs substantially: we can inject arbitrarily complex First-Order logic rules rather than just paths between two entities.
%

%
In \cite{DBLP:conf/emnlp/ChangYYM14,DBLP:conf/dsaa/KrompassNT14,DBLP:conf/semweb/KrompassBT15}, authors make use of type information about entities for only considering interactions between entities belonging to the domain and range of each predicate, assuming that type information about entities is complete.
%
In \cite{sac16}, authors assume that type information can be incomplete, and propose to adaptively decrease the score of each missing triple depending on the available type information.
%
These works consider type information about entities: in this work we propose a method for leveraging equivalence and inversion axioms, which can be used jointly with the aforementioned methods.
%

%
In \cite{DBLP:conf/kdd/0001GHHLMSSZ14,DBLP:conf/nips/NickelJT14,DBLP:conf/ijcai/WangWG15}, authors propose combining observable patterns in the form of rules and latent features for link prediction tasks.
%
However, in such models, rules are not used \emph{during} the parameters learning process, but rather \emph{after}, in an ensemble fashion.
%

%
The work in this paper is related to the \mdl{Model FSL} proposed in~\cite{DBLP:conf/emnlp/DemeesterRR16}: they use simple rules in the form of Eq.~\ref{eq:simple} for defining a partial ordering in the relation embeddings for a variant of \mdl{Model F}~\cite{DBLP:conf/naacl/RiedelYMM13}.
%
Their approach is \emph{lifted} since they do not rely on entity embeddings for satisfying the rules.
%
\ARI extends and improves over \mdl{Model FSL} -- in particular:
%
\begin{inparaenum}[1)]
%
 \item It can be used for injecting arbitrarily complex First-Order logic rules as in Eq.~\ref{eq:rule}, and
%
 \item It can be used jointly with any Knowledge Graph embedding model that provides a fact scoring function $\fscore(\Cdot ; \params)$.
%
\end{inparaenum}
%
It also improves over the work in \cite{DBLP:conf/naacl/RocktaschelSR15}, since it does not need to generate all possible instantiations of each rule.
%